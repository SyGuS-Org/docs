\section{Participating Benchmarks}
\label{sec:benchs}

In addition to last year's benchmarks, we received 4 new sets of benchmarks this year, which are shown in Table~\ref{tbl:new-benchmarks}. 


\paragraph{Program Repair}
The 18 program repair benchmarks correspond to the task of generating small expression repairs that are consistent with a given set of input-output examples~\cite{repairbenchmarks}. These benchmarks were extracted from real-world Java bugs by manually analyzing the developer commits that involved changes to fewer than 5 lines of code. The key idea of the program repair approach is to the first localize the fault location in a buggy program and generate the corresponding input-output example behavior for the buggy expression from passing test cases. In the second phase, the task of repairing the buggy expression can be framed as a SyGuS problem, where the goal is to synthesize an expression that comes from a family of expressions defined using a context-free grammar of expressions and that satisfies the input-output example constraints.

\paragraph{Crypto Circuits}
The Crypto Circuits benchmarks comprise of tasks of synthesizing constant-time circuits that are cryptographically resilient to timing attacks~\cite{EldibWW16}. Consider a circuit $C$ with a set of \emph{private} inputs $I_0$ and a set of \emph{public} inputs $I_1$ such that if an attacker changes the values of
the public inputs and observes the corresponding output, she is unable to infer the values
of the private inputs (under standard assumptions about computational resources in cryptography). An attacker can gain information about private inputs by analyzing the time the circuit takes to compute the output values on public inputs, e.g. when a public input bit changes from 1 to 0, a specific output bit is guaranteed to
change from 1 to 0 independent of whether a particular private input bit is 0 or 1, but 
may change faster when this private input is 0, thus leaking information.
The timing attack can be prevented if the circuit satisfies the \emph{constant-time} property:
A constant-time circuit is the one in which the length of all input-to-output paths  measured in terms of number of gates
are the same.

The problem of synthesizing a new circuit $C'$ that is functionally equivalent to a given circuit $C$ such that $C'$ is a constant-time circuit can be formalized as a SyGuS problem. A context-free grammar can be used to define the set of all constant-time circuits with all input-to-output path lengths within a given bound, and the functional equivalence constraint can be expressed as a Boolean formula~\cite{EldibWW16}.

\paragraph{Instruction Selection}
The Instruction Selection benchmarks consist of tasks for synthesizing a ``Bit Test and Reset" instruction from the set of basic bitvector operations, in a way similar to the implementations supported by the x86 processors. These benchmarks comprise of 4 different addressing variants with increasing levels of complexity:
\begin{itemize}
	\item btr*: Read from register.
	\item btr-am-base*: Load from memory address base.
	\item btr-am-base-index*: Load from memory address base with indexing.
	\item btr-am-base-index-scale-disp*:  Load from memory address base with index shifted with scale.
\end{itemize}

\paragraph{Invariant Generation}
The invariant generation benchmarks comprise of the task of generating a loop invariant (as a conditional linear arithmetic expression) given the pre-condition, post-condition and the transition function corresponding to the loop body. The 7 new benchmarks~\cite{PadhiM17} correspond to loop invariant tasks adapted from several recent invariant inference papers including generating path invariants, abductive inference, and NECLA Static analysis benchmarks.


\begin{table}
	\small
	\begin{center}
	\scalebox{0.94}{
		\begin{tabular}{rcl}
			Benchmark Set &  \# of benchmarks &  Contributors \\ \hline \hline
			Invariant Generation & 7 & Saswat Padhi (UCLA)  \\
			Program Repair & 18 & 	Xuan Bach D Le (SMU), David Lo (SMU) and Claire Le Goues (CMU) \\
			Crypto Circuits & 214 & Chao Wang (USC) \\		
			Instruction Selection & 28 & Sebastian Buchwald (KIT) and Andreas Fried (KIT) \\
		\end{tabular}
	}
	\end{center}
	\caption{New Contributed Benchmarks}
	\label{tbl:new-benchmarks}
\end{table}