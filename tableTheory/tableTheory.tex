\pdfoutput=1

\documentclass[english,a4paper,10pt]{article}

\usepackage[utf8]{inputenc}
\usepackage[T1]{fontenc}
\usepackage[english]{babel}

\usepackage{amsmath}
\usepackage{amssymb}
\usepackage{stmaryrd}
\usepackage{xargs}[2008/03/08]
\usepackage{xspace}
\usepackage[unicode=true,
            pdfusetitle,
            bookmarks=true,
            bookmarksnumbered=false,
            bookmarksopen=false,
            breaklinks=false,
            pdfborder={0 0 0},
            backref=false,
            colorlinks=true,
            citecolor=blue]
 {hyperref}

%%%%%%%%%%%%%%%%%%%%%%%%%%%%%% LyX specific LaTeX commands.
%% For printing a cirumflex inside a formula
\newcommand{\mathcircumflex}[0]{\mbox{\^{}}}


%%%%%%%%%%%%%%%%%%%%%%%%%%%%%% User specified LaTeX commands.
\usepackage{bussproofs}
\usepackage{centernot}
\usepackage{datetime}
\usepackage{filecontents}
\usepackage{fullpage}
\usepackage{lmodern}
\usepackage{tikz}
\usepackage{pgfplots}
\usepackage{comment}
\usepackage{amsthm}
\usepackage[noabbrev,capitalize,nameinlink]{cleveref}

\usepackage{enumitem}
\setenumerate{itemsep=0.25em,topsep=0.25em,leftmargin=2em}
\setitemize{itemsep=0.25em,topsep=0.25em,leftmargin=2em}

\usetikzlibrary{automata}
\usetikzlibrary{backgrounds}
\usetikzlibrary{fit}
\usetikzlibrary{positioning}

\usepackage{listings}
\renewcommand{\lstlistingname}{Listing}

\begin{document}

\title{Proposal for a Theory of Tables}

\author{Andrew Reynolds \and Chenglong Wang}

\maketitle


\newcommand{\rem}[1]{\textcolor{red}{[#1]}}
\newcommand{\ajr}[1]{\rem{#1 --ajr}}

\global\long\def\autobox#1{#1}


\global\long\def\opname#1{\autobox{\operatorname{#1}}}


\global\long\def\dontcare{\_}


\global\long\def\bbracket#1{\left\llbracket #1\right\rrbracket }


\global\long\def\cnot#1{\centernot#1}






\global\long\def\roset#1{\left\{  #1\right\}  }


\global\long\def\ruset#1#2{\roset{#1\;\middle\vert\;#2}}


\global\long\def\union#1#2{#1\cup#2}


\global\long\def\bigunion#1#2{\bigcup_{#1}#2}


\global\long\def\intersection#1#2{#1\cap#2}


\global\long\def\bigintersection#1#2{\bigcap_{#1}#2}


\global\long\def\bigland#1#2{\bigwedge_{#1}#2}


\global\long\def\powerset#1{2^{#1}}




\global\long\def\cart#1#2{#1\times#2}


\global\long\def\tuple#1{\left(#1\right)}


\global\long\def\quoset#1#2{\left.#1\middle/#2\right.}


\global\long\def\equivclass#1{\left[#1\right]}


\global\long\def\refltransclosure#1{#1^{*}}




\newcommandx\funcapphidden[3][usedefault, addprefix=\global, 1=]{#2#1#3}


\global\long\def\funcapplambda#1#2{\funcapphidden[\,]{#1}{#2}}


\global\long\def\funcapptrad#1#2{\funcapphidden{#1}{\tuple{#2}}}


\global\long\def\funccomp#1#2{#1\circ#2}


\global\long\def\arrow#1#2{#1\to#2}


\global\long\def\func#1#2#3{#1:\arrow{#2}{#3}}




\global\long\def\N{\mathbb{N}}


\global\long\def\Z{\mathbb{Z}}


\global\long\def\Q{\mathbb{Q}}


\global\long\def\R{\mathbb{R}}


\global\long\def\D{\mathbb{D}}


\global\long\def\Zmod#1{\quoset{\Z}{#1\Z}}


\global\long\def\vector#1{\mathbf{#1}}


\global\long\def\dotprod#1#2{#1\cdot#2}


\global\long\def\bool{{\tt bool}}


\global\long\def\true{{\tt true}}


\global\long\def\false{{\tt false}}






\global\long\def\strempty{\epsilon}




\global\long\def\kstar#1{#1^{*}}
\global\long\def\koption#1{#1^{?}}
\global\long\def\kstarn#1#2{#1^{#2}}


\global\long\def\kplus#1{#1^{+}}


\global\long\def\paren#1{{\tt (}#1{\tt )}}



\section{Theory of Tables}

This document describes an SMT-LIB theory of tables $\thtable$.
We provide its signature $\sigtable$ in two parts,
first describing its sorts and then its function symbols.
We assume that $\sigtable$ includes the symbols of the core theory, 
namely, the Boolean sort denoted $\syntax{Bool}$,
a binary polymorphic predicate $\syntax{=}$ denoting equality, for all sorts.

\section{Sort Symbols}

The signature $\sigtable$ contains 
the integer sort $\syntax{Int}$ as well as
the following three (parametric) sorts.

\subsection{Function Sorts}
The signature $\sigtable$ includes all sorts of the form:
\begin{align*}
\syntax{(->\ T_1\ ...\ T_k\ T)}
\end{align*}
for all sorts $\syntax{T_1, ..., T_k, T}$ where $k \geq 1$.
This sort denotes the class of functions taking arguments of sorts $\syntax{T_1, ..., T_k}$
and having return sort $\syntax{T}$.
Note that a predicate sort is a function sort whose return
sort is Boolean.

\begin{comment}
\paragraph{Notes}
\begin{itemize}
\item The term $\syntax{ 
(lambda\ ((x_1\ T_1)\ ...\ (x_n\ T_n))\ t)
}$
has sort 
$\syntax{(->\ T_1\ ...\ T_n\ T)}$
when $\syntax{t}$ has sort $\syntax{T}$.
\end{itemize}
\end{comment}

\subsection{Tuple Sorts}

The signature $\sigtable$ includes all sorts of the form:
\begin{align*}
\syntax{ (Tuple\ T_1\ ...\ T_k) }
\end{align*}
for all sorts $\syntax{T_1, ..., T_k}$.
This sort denotes tuples taking arguments of sorts $\syntax{T_1, ..., T_k}$.
Furthermore, the sort $\syntax{Tuple}$ (with no arguments) denotes
the empty tuple.

\subsection{Bag Sorts}
The signature $\sigtable$ includes all sorts of the form:
\begin{align*}
\syntax{ (Bag\ T) }
\end{align*}
for all sorts $\syntax{T}$. 
This sort denotes bags (i.e. multisets) of elements of sort $\syntax{T}$.
Note that bags allow duplication of elements and do not maintain an order
of the elements they contain.
The \emph{cardinality} of a bag is the number of elements it contains.
%An element may occur with any multiplicity in a bag.
%The elements of a bag are unordered.

\paragraph{Notes}
A table is a bag whose elements are tuples.
We write $\syntax{ (Table\ T_1 ...\ T_k)}$ as shorthand for 
the sort $\syntax{ (Bag\ (Tuple\ T_1\ ...\ T_k)) }$ 
in the remainder of the document.
We refer to $\syntax{T_1}, \ldots, \syntax{T_k}$ as the \emph{columns} of the table sort above,
and the elements of a given table as its \emph{rows}.
Notice that the columns of the table are \emph{not} given explicit names.
The sort $\syntax{Table}$ with no arguments denotes the table sort with
no columns.

\section{Function Symbols}

The signature $\sigtable$ includes function symbols for
lambda abstractions,
tuples, bags, and
three function symbols denoting table operations.
We list these in the following.
Function symbols in its signature may be polymorphic.
To provide the sort of (classes of) polymorphic functions,
we follow the convention for theory definitions in SMT-LIB. %~\cite{}.
In particular, the \emph{sort definition} for a polymorphic function $\syntax{f}$
has the following syntax:
\begin{verbatim}
(par (X_1 ... X_j) (f T_1 ... T_k T))
\end{verbatim}
Above, $\syntax{X_1\ ...\ X_j}$ are \emph{sort parameters}
that may freely occur in sorts $\syntax{T_1,\ ...\ T_k}$ and $\syntax{T}$.
Above, $\syntax{T_1,\ ...\ T_k}$ are the argument sorts of $\syntax{f}$
and $\syntax{T}$ is its return sort.

\subsection{Functions}
We assume
the symbol $\syntax{lambda}$ denotes lambda abstraction.
Its syntax is:
\begin{align*}
\syntax{ 
(lambda\ ((x_1\ T_1)\ ...\ (x_n\ T_n))\ t)
}
\end{align*}
where $\syntax{t}$ is a term,
and $\syntax{x_1\ ...\ x_n}$ are bound variables.
It has sort $\syntax{(->\ T_1\ ...\ T_n\ T)}$
where $\syntax{t}$ has sort $\syntax{T}$.
It denotes the anonymous function
returning $t$ for input arguments $\syntax{x_1}, \ldots, \syntax{x_n}$.
%The term 
%$\syntax{((lambda\ ((x_1\ T_1)\ ...\ (x_n\ T_n))\ t)\ s_1\ ...\ s_n)}$
%is interpreted as $t$ where $x_1, \ldots, x_n$ are replaced by
%$s_1, \ldots, s_n$.

\subsection{Tuples}

\subsubsection{Tuple Construction}
The signature $\sigtable$ includes 
the function symbol $\syntax{tuple}$.
Its sort definition is
\begin{verbatim}
(par (X_1 ... X_j) 
  (tuple
  
    ; The arguments of the tuple we are constructing
    X_1 ... X_j
    
    ; The return sort
    (Tuple X_1 ... X_j)
  )
)
\end{verbatim}
The term $\syntax{(tuple\ u_1\ ...\ u_n)}$ is
interpreted as the tuple comprised of 
the (ordered) list of elements $\syntax{u_1,\ ...\ u_n}$.
\paragraph{Notes}
\begin{itemize}
\item
The list of arguments provided to the tuple constructor can be empty.
The term $\syntax{tuple}$, when applied to no arguments, denotes the empty tuple,
i.e. the tuple with zero components.
\end{itemize}

\subsubsection{Tuple Selection}
The signature $\sigtable$ includes 
the polymorphic function symbol $\syntax{tuple.select}$.
Its sort definition is parametrized on $j$ sorts:
\begin{verbatim}
(par (X_1 ... X_j) 
  (tuple.select
    
    ; The index we are selecting 
    Int
    
    ; The tuple we are selecting from
    (Tuple X_1 ... X_j)
    
    ; The return sort
    X_n
  )
)
\end{verbatim}
When $1 \leq n \leq j$,
the term $\syntax{(tuple.select\ n\ t)}$ is
interpreted as the $\syntax{n}^{th}$ argument of the tuple $\syntax{t}$.
When $n$ is out of bounds, e.g. $1 \leq n \leq j$ does not hold,
then the value of $\syntax{(tuple.select\ n\ t)}$ is unspecified.

\paragraph{Notes}
\begin{itemize}
\item Like other operators with partially specified semantics,
the symbol $\syntax{tuple.select}$ behaves like a function, meaning
that $\syntax{(tuple.select\ n_1\ t_1)}$ and  $\syntax{(tuple.select\ n_2\ t_2)}$
must have the same value when $n_1, t_1$ pairwise have the same value as $n_2, t_2$,
even when $n_1$ and $n_2$ are out of bounds.
\item When $n$ is a numeral such that $1 \leq n \leq j$,
$\syntax{((\_\ tuple.select\ n)\ t)}$ is syntax sugar for the term
$\syntax{(tuple.select\ n\ t)}$.
The former may be used to make it is explicit the index to a tuple selector
is within bounds, and is not allowed to be symbolic.
\end{itemize}

\subsubsection{Tuple Projection}
\label{sec:tup-project}
The signature $\sigtable$ includes 
the function symbol $\syntax{tuple.project}$.
Its sort definition includes an unbounded number of integer arguments, followed
by a tuple:
\begin{verbatim}
(par (X_1 ... X_j) 
  (tuple.project
    
    ; the components to project
    Int ... Int
  
    ; The table to project
    (Tuple X_1 ... X_j)
    
    ; The return sort
    (Tuple X_{n_1} ... X_{n_k})
  )
)
\end{verbatim}
The term
$\syntax{(tuple.project\ n_1\ ...\ n_k\ t)}$
returns the tuple obtained by collecting the
values from $t$ at positions $n_1 \ldots n_k$.
In other words, this term is equivalent to:
\begin{align*}
\syntax{(tuple\ (tuple.select\ n_1\ t)\ ...\ (tuple.select\ n_k\ t))}
\end{align*}
Notice that if any $n_i$ in the above term is out of bounds, the value of
that component of the returned tuple is unspecified.
%If for each $i = 1, \ldots, k$,
%$1 \leq n_i \leq j$,

\paragraph{Notes}
\begin{itemize}
\item
When $k=0$,
we have that $\syntax{(tuple.project\ t)}$ is equivalent to the empty tuple.
\item
The indices $n_1, \ldots, n_k$ do not need to be ordered such that $n_1 < \ldots < n_k$.
Moreover, it may be the case that $n_i = n_j$ for some $i \neq j$ in which
case the component of the original is duplicated in the return value of the projection.
\item
The list of integer arguments to tuple projection can be empty.
The function application
$\syntax{(tuple.project\ t)}$ always returns the empty tuple for any input $t$.
\item Similar to the indexed syntax for tuple selection,
when $n_1, \ldots, n_k$ are numerals in the range $[1, j]$,
$\syntax{((\_\ tuple.project\ n_1\ ...\ n_k)\ t)}$ is syntax sugar for the term
$\syntax{(tuple.project\ n_1\ ...\ n_k\ t)}$.
\end{itemize}

\subsection{Bags}

\subsubsection{Bag Construction}
The signature $\sigtable$ includes 
the function symbol $\syntax{bag}$ whose sort definition is
\begin{verbatim}
(par (X) 
  (bag
  
    ; The argument and multiplicity of the bag we are constructing
    X Int
    
    ; The return sort
    (Bag X)
  )
)
\end{verbatim}
The term $\syntax{(bag\ t\ n)}$ is
interpreted as the bag containing $n$ copies of the element $\syntax{t}$.
%where notice the order of these elements is irrelevant
%and $\syntax{t_i}$ may be equal to $\syntax{t_j}$ for any $\syntax{i},\syntax{j}$.
%If $n=0$,
%then $\syntax{bag}$ denotes the empty bag.
Notice that bags containing multiple distinct elements
can be constructed using $\syntax{bag.union\_disjoint}$,
introduced in Section~\ref{sec:bag-compositions}

\subsubsection{Bag Count}
The signature $\sigtable$ includes 
the function symbol $\syntax{bag.count}$ whose sort definition is
\begin{verbatim}
(par (X)
  (bag.count 
  
    ; The element we are counting
    X 
    
    ; The bag in which the element may occur
    (Bag X)
    
    ; The return sort
    Int
  )
)
\end{verbatim}
The term $\syntax{(count\ u\ t)}$ is
interpreted as the number of times element $\syntax{u}$ occurs in $\syntax{t}$.

\paragraph{Notes}
\begin{itemize}
\item
We do not include an explicit membership predicate for bags, although note that
the formula $\syntax{(>\ (bag.count\ u\ t)\ 0)}$ holds exactly when $\syntax{u}$
is a member of $\syntax{t}$ (with any multiplicity).
\end{itemize}

\subsubsection{Bag Compositions}
\label{sec:bag-compositions}
The signature $\sigtable$ includes 
the binary function symbols 
$\syntax{bag.union\_max}$, $\syntax{bag.union\_disjoint}$, 
$\syntax{bag.intersect\_min}$, $\syntax{bag.difference\_subtract}$ and
$\syntax{bag.difference\_remove}$.
Each $\syntax{f}$ of the aforementioned operators
has sort definition:
\begin{verbatim}
(par (X) (f (Bag X) (Bag X) (Bag X)))
\end{verbatim}
These functions are such that for all elements $\syntax{x}$
and bags $\syntax{A,B}$:
\begin{verbatim}
(bag.count x (bag.union_max A B))            = (max (bag.count x A) (bag.count x B))
(bag.count x (bag.union_disjoint A B))       = (+ (bag.count x A) (bag.count x B))
(bag.count x (bag.intersect_min A B))        = (min (bag.count x A) (bag.count x B))
(bag.count x (bag.difference_subtract A B))  = (max (- (bag.count x A) (bag.count x B)) 0)
(bag.count x (bag.difference_remove A B))    = (ite (> (bag.count x B) 0) 0 (bag.count x A))
\end{verbatim}
where $\syntax{max}$ and $\syntax{min}$ are binary functions
returning the maximum (resp. minimum) of two integers, and $\syntax{ite}$
is the if-then-else construct.

\paragraph{Notes}
\begin{itemize}
\item
The above operators are \emph{chainable}, that is,
$\syntax{(bag.union_max\ A\ B\ C)}$ is syntax sugar for
$\syntax{(bag.union_max\ (bag.union_max\ A\ B)\ C)}$.
\end{itemize}

\subsubsection{Bag Cardinality}
The signature $\sigtable$ includes 
the function symbol $\syntax{card}$ whose sort definition is
\begin{verbatim}
(par (X) 
  (bag.card 
  
    ; The bag to process
    (Bag X) 
    
    ; The return sort
    Int
  )
)
\end{verbatim}
The term $\syntax{(bag.card\ t)}$ is equal to
the total number of elements (taking into account multiplicity)
that occur in the bag $\syntax{t}$.
For example, the term $\syntax{(bag.card\ (bag\ t\ n))}$ is
equivalent to $\syntax{n}$.

\subsubsection{Bag Filtering}
The signature $\sigtable$ includes 
the function symbol $\syntax{bag.filter}$. Its sort definition is:

\begin{verbatim}
(par (X)
  (bag.filter
  
    ; The predicate we are filtering based on
    (-> X Bool)
  
    ; The bag to filter
    (Bag X)
    
    ; The return sort
    (Bag X)
  )
)
\end{verbatim}

\paragraph{Semantics}
The filter operator keeps only the elements
from its first argument
for which the predicate of its second argument is true.
In other words,
$\syntax{(bag.filter\ P\ t)}$
is the bag containing exactly
the elements $u$ from $\syntax{t}$
such that $\syntax{(P\ u)}$ is true.
Notice that if $\syntax{(P\ u)}$ is true,
then $u$ has the same multiplicity in $\syntax{(bag.filter\ P\ t)}$
as it did in $\syntax{t}$.

\subsubsection{Bag Map}
The signature $\sigtable$ includes 
the function symbol $\syntax{map}$. Its sort definition is:

\begin{verbatim}
(par (X_1 X_2)
  (bag.map
    
    ; The function we are applying
    (-> X_1 X_2)
  
    ; The bag to process
    (Bag X_1)
    
    ; The return sort
    (Bag X_2)
  )
)
\end{verbatim}

\paragraph{Semantics}
The map operator runs the function provided
in the second argument to all elements in the bag.
In other words,
$\syntax{(bag.map\ f\ t)}$
is the bag containing exactly
the elements $\syntax{(f\ u)}$
for each $\syntax{u}$ that occurs in $\syntax{t}$.
The multiplicity of $\syntax{(f\ u)}$
in $\syntax{(bag.map\ f\ t)}$
is the same as the multiplicity of $\syntax{u}$ in $\syntax{t}$.

\subsubsection{Bag Fold}
\label{sec:fold}
The signature $\sigtable$ includes 
the function symbol $\syntax{bag.fold}$. Its sort definition is:

\begin{verbatim}
(par (X_1 X_2)
  (bag.fold
    
    ; The "combining" function we are applying
    (-> X_1 X_2 X_2)
    
    ; The initial value of the fold
    X_2
  
    ; The bag to process
    (Bag X_1)
    
    ; The return sort
    X_2
  )
)
\end{verbatim}


\paragraph{Semantics}
Given a bag $t$ whose elements are $\syntax{u_1}, \ldots, \syntax{u_n}$,
assume an arbitrary total ordering over these elements $\preceq$.
Then, $\syntax{(bag.fold\ c\ i\ t)}$ is equivalent to
$
\syntax{
(c\ u_n\ ...\ (c\ u_2\ (c\ u_1\ i))\ ...\ )
}
$.
In other words, given a start value $\syntax{i}$,
we apply the combining function $\syntax{c}$, taking these elements
as first arguments in order.

\paragraph{Notes}
\begin{itemize}
\item
The semantics above impose no restrictions on the ordering $\preceq$,
which determines the order in which the tuples $\syntax{u_1}, \ldots, \syntax{u_n}$
are passed to the combining function $\syntax{c}$.
Moreover, 
we do not provide syntax to specify such an ordering.
Instead, 
we remark that the above semantics
do not require an ordering when the combining function is
\emph{order-agnostic},
that is, for all $\syntax{i}, \syntax{v_1}, \ldots, \syntax{v_n}$,
we have that:
\begin{eqnarray*}
\syntax{(c\ v_1\ ...\ (c\ v_n\ i)\ ...\ )\ =\ (c\ w_1\ ...\ (c\ w_n\ i)\ ...\ )}
\end{eqnarray*}
for all permutations $( \syntax{w_1}, \ldots, \syntax{w_n} )$ of $( \syntax{v_1}, \ldots, \syntax{v_n} )$.
If the function $\syntax{c}$ is order-agnostic,
then the result of the fold in the above computation will be the same regardless
of the ordering $\preceq$.
Thus, as stated, the above semantics assume that $\syntax{c}$ is order-agnostic.
The semantics
of fold over combining functions that are not order-agnostic are undefined.
%Thus, we assume that the operator $c$ is
%associatiative and commutative.
\end{itemize}

\subsubsection{Bag Partition}
The signature $\sigtable$ includes 
the function symbol $\syntax{bag.partition}$. Its sort definition is:

\begin{verbatim}
(par (X)
  (bag.partition
    
    ; The equivalence relation
    (-> X X Bool)
  
    ; The bag to process
    (Bag X)
    
    ; The return sort
    (Bag (Bag X))
  )
)
\end{verbatim}

\paragraph{Semantics}
Given a bag $t$, 
the term $\syntax{(bag.partition\ r\ t)}$ is interpreted
as a bag of bags that partition the elements of $\syntax{t}$,
whose elements (of bag sort) are $\syntax{t_1}, \ldots, \syntax{t_n}$.
For each pair of elements $\syntax{u_i}, \syntax{u_j}$
occurring in $\syntax{t_k}, \syntax{t_l}$ respectively,
we have that $k=l$ if and only if $\syntax{(r\ u_i\ u_j)}$ is true.

\paragraph{Notes}
\begin{itemize}
\item
The argument $r$ passed as the
first argument to $\syntax{bag.partition}$ is expected to be an equivalence relation.
In particular, this means it is reflexive, symmetric and transitive.
This ensures that a partition with the above properties
can be computed.
The semantics of applications of $\syntax{bag.partition}$ for relations $\syntax{r}$
that are not equivalence relations is undefined.
\item
If $\syntax{u}$ occurs with some multiplicity $n$ in $\syntax{t}$,
then $\syntax{u}$ occurs with the same multiplicity $n$ in some $\syntax{t_i}$.
\end{itemize}

\subsection{Tables}

\subsubsection{Table Projection}
The signature $\sigtable$ includes 
the functions symbol of the form
$\syntax{tuple.project}$.
%Note this operator is overloaded,
%as a separate projection function (of the same name)
%is also defined for tuples in Section~\ref{sec:tup-project}.
Its sort definition includes an unbounded number of integer arguments, followed
by a tuple:
\begin{verbatim}
(par (X_1 ... X_j) 
  (table.project
  
    ; The columns to project
    Int ... Int
  
    ; The table to project
    (Table X_1 ... X_j)
    
    ; The return sort
    (Table X_{n_1} ... X_{n_k})
  )
)
\end{verbatim}
%where for each $i = 1, \ldots, j$,
%$1 \leq n_i \leq j$.

\paragraph{Semantics}

A projection operator for tables keeps (copies) of the given columns
in each (tuple) element of its argument.
In other words,
consider the term $\syntax{(table.project\ n_1\ ...\ n_k\ t)}$
where $t$ is a table and $n_1, \ldots, n_k$ are integers.
%The value of this term is the table containing a tuple of the form
%$(u_{n_1}, \ldots, u_{n_k})$
%for each tuple $(u_1, \ldots, u_j)$ from $t$.
%More precisely,
%the term $\syntax{((table.project\ n_1\ ...\ n_k\ t)}$
This term
can be seen as syntax sugar for $\syntax{(map\ p\ t)}$ where $\syntax{p}$
is the function:
\begin{verbatim}
(lambda ((u (Tuple X_1 ... X_j))) (tuple.project n_1 ... n_k u))
\end{verbatim}
In other words, table projection
applies the tuple projection on all tuples in its table argument $\syntax{t}$
to obtain tuples in the returned table.

\paragraph{Notes}
\begin{itemize}
\item
Like tuple projection,
it may be the case that $\syntax{n_i} = \syntax{n_j}$ for some $i \neq j$,
in which case two (or more) copies of the column at that position
are in the result of the projection.
\item
The cardinality of $\syntax{(table.project\ n_1\ ...\ n_k\ t)}$ is
the same as the cardinality of $\syntax{t}$.
\item
For any table $t$,
the application taking no integer arguments $\syntax{(table.project\ t)}$ 
returns a table with no columns whose cardinality is the same as that of $\syntax{t}$.
\item
When $k>0$ and
$n_1, \ldots, n_k$ are numerals in the range $[1, j]$,
$\syntax{((\_\ table.project\ n_1\ ...\ n_k)\ t)}$ is syntax sugar for the term
$\syntax{(table.project\ n_1\ ...\ n_k\ t)}$.
\end{itemize}

\subsubsection{Table Product}
The signature $\sigtable$ includes 
the function symbol $\syntax{table.product}$
%where $n_1, m_1, \ldots, n_k, m_k$ are numerals.
Its sort definition is:

\begin{verbatim}
(par (X_1 ... X_j Y_1 ... Y_l)
  (table.product
  
    ; The tables to take the product of
    (Table X_1 ... X_j)
    (Table Y_1 ... Y_j)

    ; The return sort
    (Table X_1 ... X_j Y_1 ... Y_j)
  )
)
\end{verbatim}
\paragraph{Semantics}
We have that $\syntax{(table.product\ t_1\ t_2)}$
denotes the table containing a tuple corresponding to the concatentation
of tuples $u_1$ and $u_2$
for each $u_1 \in \syntax{t_1}$ and $u_2 \in \syntax{t_2}$.

\subsubsection{Table Join}
The signature $\sigtable$ includes 
the function symbol $\syntax{table.join}$
%where $n_1, m_1, \ldots, n_k, m_k$ are numerals.
Its sort definition is:

\begin{verbatim}
(par (X_1 ... X_j Y_1 ... Y_l)
  (table.join
    
    ; the pairs of columns to join
    Int Int ... Int Int
  
    ; The tables to join
    (Table X_1 ... X_j)
    (Table Y_1 ... Y_j)
    
    ; The return sort
    (Table X_1 ... X_j Y_1 ... Y_j)
  )
)
\end{verbatim}
%where for all $i = 1, \ldots k$,
%we have that $1 \leq \syntax{n_i} \leq \syntax{j}$, $1 \leq \syntax{m_i} \leq \syntax{l}$
%and $X_{n_i}$ is the same type as $Y_{m_i}$.
%Note that
%$\syntax{join}$ (without numeral indices) denotes the above operator where $\syntax{k}$ is $0$.

\paragraph{Semantics}
We have that $\syntax{(table.join\ n_1\ m_1\ ...\ n_k\ m_k\ t_1\ t_2)}$
is syntax sugar for $\syntax{(filter\ P\ (table.product\ t_1\ t_2))}$ where $\syntax{P}$ is
the predicate:
\begin{verbatim}
(lambda ((u1 (Tuple X_1 ... X_j)) (u2 (Tuple Y_1 ... Y_l)))
  (= (tuple.project n_1 ... n_k u1) (tuple.project m_1 ... m_k u2))
)
\end{verbatim}
In other words,
the join keeps only concatenations of tuples $u_1 \in \syntax{t_1}$ and
$u_2 \in \syntax{t_2}$ 
that match on each of the pairs of argument indices (columns) specified by
the given indices,
where $n_1\ ...\ n_k$ are indices corresponding to arguments of $u_1$
and $m_1\ ...\ m_k$ are indices corresponding to arguments of $u_2$.
Note that if $k=0$, all concenations are kept, and the join is
equivalent to a product.

\paragraph{Notes}
\begin{itemize}
\item
When the join operator takes no integer arguments,
we have that $\syntax{(table.join\ t_1\ t_2)}$
is equivalent to $\syntax{(table.product\ t_1\ t_2)}$.
\item
In common usage,
the indices provided to this operator specify columns that are in bounds.
When the indices are out of bounds,
the semantics of the join are the same as above, 
which however rely on the semantics of tuple projection for out of bound indices.
\item 
The cardinality of $\syntax{(table.join\ t_1\ t_2)}$
is the cardinality of $\syntax{t_1}$ times the cardinality of $\syntax{t_2}$.
\item
If $k>0$ and $n_1, \ldots, n_k$ are numerals in the range $[1, j]$ 
and $m_1, \ldots, m_k$ are numerals in the range $[1, l]$, then
$\syntax{((\_\ table.join\ n_1\ m_1\ ...\ n_k\ m_k)\ t_1\ t_2)}$
is syntax sugar for the term
$\syntax{(table.join\ n_1\ m_1\ ...\ n_k\ m_k\ t_1\ t_2)}$.
\end{itemize}


\subsubsection{Table Aggregation}
The signature $\sigtable$ includes 
the function symbol $\syntax{table.aggr}$
called \emph{table aggregation operator}. Its sort definition is:

\begin{verbatim}
(par (X_1 ... X_j X) 
  (table.aggr
  
    ; The columns to aggregate
    Int ... Int
    
    ; The "combining function"
    (-> (Tuple X_1 ... X_j) X X)
    
    ; The "initial value"
    X
    
    ; The table to aggregate
    (Table X_1 ... X_j)
    
    ; The return sort
    (Bag X)
  )
)
\end{verbatim}

\paragraph{Semantics}
Let $\syntax{(table.aggr\ n_1\ ...\ n_k\ c\ i\ t)}$
be a well-sorted term for
combining function $\syntax{c}$, 
initial value $\syntax{i}$,
and table-to-aggregate $\syntax{t}$.
Assume that $\syntax{c}$ is \emph{order-agnostic} (see Section~\ref{sec:fold}).
The application $\syntax{(table.aggr\ n_1\ ...\ n_k\ c\ i\ t)}$ is
the union of elements computed via a fold operation involving $\syntax{c}$ and $\syntax{i}$
that run over elements in a partition of the table $\syntax{t}$.
In particular, this application of an aggregation function is syntax sugar for:
\begin{verbatim}
(bag.map (bag.fold c i) (bag.partition eq t))
\end{verbatim}
where $\syntax{eq}$ is the relation over tuples that holds
when tuples are equivalent for indices $\syntax{n_1}, \ldots, \syntax{n_k}$:
\begin{verbatim}
(lambda ((t_1 (Tuple X_1 ... X_j)) (t_2 (Tuple X_1 ... X_j)))
  (= (tuple.project n_1 ... n_k t_1) (tuple.project n_1 ... n_k t_2))
)
\end{verbatim}
Above, notice that 
$\syntax{(bag.fold\ c\ i)}$ is a partial application of the fold
function which takes a $\syntax{(Bag\ (Tuple\ X_1\ ...\ X_j))}$
and returns a term of type $\syntax{X}$.

\paragraph{Notes}
\begin{itemize}
\item
In common usage,
the indices provided to table aggregation specify columns that are in bounds.
When the indices are out of bounds,
the semantics are the same as above, 
which however rely on the semantics of tuple projection for out of bound indices.
\item
When $n_1, \ldots, n_k$ are numerals in the range $[1,j]$,
we have that
$\syntax{((\_\ table.aggr\ n_1\ ...\ n_k)\ c\ i\ t)}$
is syntax sugar for the term
$\syntax{(table.aggr\ n_1\ ...\ n_k\ c\ i\ t)}$.
\end{itemize}

\begin{comment}
In detail,
let $\{ t_1, \ldots, t_n \}$ be a partition of table $t$
where for each table $t_i$, each of its (tuple) elements
are identical for arguments $n_1, \ldots, n_k$.
%In other words, 
%all elements of $\syntax{((\_\ project\ n_1\ ...\ n_k) t_i)}$
%are identical.
For each $i = 1, \ldots, n$,
let $M_i$ be the bag $p(t_i)$
where $p$ is the projection operator denoted by $\syntax{(\_\ project\ n)}$.
For each such bag $M_i = \{ u_{i1}, \ldots, u_{in} \}$,
assume an arbitrary ordering over these elements $\preceq$.
Let $r_i$ be the term:
\begin{align*}
f( M_i, c( u_{in}, \ldots, c( u_{i1}, b ) \ldots ) )
\end{align*}
where $u_{i1} \preceq \ldots \preceq u_{in}$.
%which is a term whose type is that of the $n^{th}$ column of the type of $t$.
In other words,
$r_i$ is the result of a fold operation
starting with \emph{initial value} $b$,
applying \emph{combining function} $c$ to
the accumulated value and each of the tuples $u_{i1}, \ldots, u_{in}$,
and finally applying a \emph{finalization} function $f$
to the bag $M_i$ and to the result of the aforementioned computation.
Then,
$A( t, b, c, f)$ is
the bag union of (single argument) tuples $\{ (r_1), \ldots, (r_n) \}$.

\paragraph{Unspecified Finalize Function}
We overload the aggregation function
such that its fourth argument is optional.
If the fourth argument is not specified,
then 
$
\syntax{
((\_\ aggr\ n_1\ ...\ n_k\ n)\ t\ b\ c)
}
$
is syntax sugar for
$
\syntax{
((\_\ aggr\ n_1\ ...\ n_k\ n)\ t\ b\ c\ ident)
}
$
where $\syntax{ident}$ is the function
$\syntax{(lambda\ ((y\ (Table\ T_1 ...\ T_k))\ (x\ X_a))\ x)}$.
In other words, this function returns the result of the
fold computation described above unchanged.
\end{comment}

\section{Extensions}

In this section, we provide an extended syntax for representing
\emph{abstract} sorts and function symbols.
%We provide an extended type system for type checking terms over these symbols.

\subsection{Abstract Sorts}
We write
$\syntax{\abstype{Tuple}}$ and $\syntax{\abstype{Bag}}$
to denote the abstract tuple type and the abstract bag type respectively.
These types take no parameters.
When a term $t$ is annotated with an abstract tuple type,
conceptually the type of $t$ is some tuple type,
but the parameters of that type is not given.
We write $\syntax{\abstype{Table}}$ as shorthand for $\syntax{( Bag\ \abstype{Tuple})}$,
that is, bags of tuples whose parameters are not given.
Furthermore, we write $\syntax{\abstypebase}$ to denote
the abstract type.
An annotation of the abstract type on a term $t$ gives no information about the type of $t$.

\section{Examples}

\subsection{Table Transformation}
The following is an example using the table theory in a syntax-guided synthesis problem
in the SyGuS 2.1 format.

\begin{verbatim}
; Predicates for tuples
(define-fun p_true ((r (Tuple Int String))) Bool true)
(define-fun p_false ((r (Tuple Int String))) Bool false)
(define-fun p_gt_zero ((r (Tuple Int String))) Bool (> ((_ tuple.select 0) r) 0))
(define-fun p_lt_zero ((r (Tuple Int String))) Bool (< ((_ tuple.select 0) r) 0))
(define-fun p_str_emp ((r (Tuple Int String))) Bool (= (str.len ((_ tuple.select 1) r)) 0))

; Synthesize a transformation for tables with an integer and string column
(synth-fun f ((x (Table Int String))) (Table Int String)(
(Start (Table Int String))
(StartTuple (Tuple Int String))
(StartInt Int)
(StartString String)
(StartPred (-> (Tuple Int String) Bool))
)(
(Start (Table Int String) (
  x
  (bag StartTuple StartInt)
  (bag.filter StartPred Start)
  (bag.union_disjoint Start Start)
))
(StartTuple (Tuple Int String) (
  (tuple StartInt StartString)
))
(StartInt Int (
  0 1
  ((_ tuple.select 0) StartTuple)
))
(StartString String (
  "" "A" "B" "C" "D" "E" 
  ((_ tuple.select 1) StartTuple)
))
(StartPred (-> (Tuple Int String) Bool)(
  p_true p_false p_gt_zero p_lt_zero p_str_emp
))
))
(constraint (= (f 
(bag.union_disjoint
(bag (tuple 1 "A") 1)
(bag (tuple (- 2) "B") 1)
(bag (tuple (- 1) "C") 1)
(bag (tuple 1 "C") 1)
(bag (tuple 0 "D") 1)
(bag (tuple 2 "E") 1))
)
(bag.union_disjoint
(bag (tuple 1 "A") 1)
(bag (tuple 1 "C") 1)
(bag (tuple 2 "E") 1))
))
(check-synth)
\end{verbatim}
In this example, we are looking to synthesize a transformation on
tables with one integer column and one string column whose body
fits the provided grammar.
A possible correct response for this example from a synthesis solver is:
\begin{verbatim}
(
(define-fun f ((x (Table Int String))) (Table Int String)
  (bag.filter p_gt_zero x))
)
\end{verbatim}
In particular, the transformation filters the rows of the table such that
only those whose integer column is greater than zero, as specified by the
predicate $\syntax{p\_gt\_zero}$ are kept.

\subsection{Table Transformation with Abstract Sorts}
The following additionally makes use of abstract sorts.

\begin{verbatim}
(synth-fun f ((x (Table String))) (Table Int String) (
(Start ?Table)
(StartTuple (Tuple Int))
)(
(Start ?Table (
  x
  (bag StartTuple StartInt)
  (bag.union_disjoint Start Start)
  (table.product Start Start)
))
(StartTuple (Tuple Int) (
  (tuple StartInt)
))
(StartInt Int (
  0 1 7
  (tuple.select StartInt Start)
))
))
(constraint (= (f 
(bag.union_disjoint
(bag (tuple "A") 1)
(bag (tuple "C") 1)
(bag (tuple "E") 1))
)
(bag.union_disjoint
(bag (tuple 7 "A") 1)
(bag (tuple 7 "C") 1)
(bag (tuple 7 "E") 1))
))
(check-synth)
\end{verbatim}
In this example, we are looking to synthesize a function that transforms
a table having a string column to one that has an integer and a string column.
Notice that the grammar for $\syntax{f}$ begins with a non-terminal
symbol $\syntax{Start}$ whose type annotation is $\syntax{?Table}$.
In particular, this means that $\syntax{Start}$ may generate terms having
\emph{any} table type. However, a term generated by this non-terminal is
only valid as a solution for $\syntax{f}$ if it has type $\syntax{(Table\ Int\ String)}$,
the return type of $\syntax{f}$.
Note that several non-terminals of the above grammar are well-typed only for a subset of the terms they apply to.
For example, $\syntax{(tuple.select\ StartInt\ Start)}$ is the right hand side of
a production rule for $\syntax{StartInt}$.
Thus, it is assumed that this rule is limited to generating terms
$\syntax{(tuple.select\ n\ s)}$ where $\syntax{s}$ is a table whose $\syntax{n}^{th}$ column is an integer.

A possible correct response for this example from a synthesis solver is:
\begin{verbatim}
(
(define-fun f ((x (Table String))) (Table Int String) 
  (table.product (bag (tuple 7) 1) x)
)
\end{verbatim}

In particular, this prepends the integer $\syntax{7}$ to all rows of the original table.

\end{document}
