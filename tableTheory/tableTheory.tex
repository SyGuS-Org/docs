\pdfoutput=1

\documentclass[english,a4paper,10pt]{article}

\usepackage[utf8]{inputenc}
\usepackage[T1]{fontenc}
\usepackage[english]{babel}

\usepackage{amsmath}
\usepackage{amssymb}
\usepackage{stmaryrd}
\usepackage{xargs}[2008/03/08]
\usepackage{xspace}
\usepackage[unicode=true,
            pdfusetitle,
            bookmarks=true,
            bookmarksnumbered=false,
            bookmarksopen=false,
            breaklinks=false,
            pdfborder={0 0 0},
            backref=false,
            colorlinks=true,
            citecolor=blue]
 {hyperref}

%%%%%%%%%%%%%%%%%%%%%%%%%%%%%% LyX specific LaTeX commands.
%% For printing a cirumflex inside a formula
\newcommand{\mathcircumflex}[0]{\mbox{\^{}}}


%%%%%%%%%%%%%%%%%%%%%%%%%%%%%% User specified LaTeX commands.
\usepackage{bussproofs}
\usepackage{centernot}
\usepackage{datetime}
\usepackage{filecontents}
\usepackage{fullpage}
\usepackage{lmodern}
\usepackage{tikz}
\usepackage{pgfplots}
\usepackage{comment}
\usepackage{amsthm}
\usepackage[noabbrev,capitalize,nameinlink]{cleveref}

\usepackage{enumitem}
\setenumerate{itemsep=0.25em,topsep=0.25em,leftmargin=2em}
\setitemize{itemsep=0.25em,topsep=0.25em,leftmargin=2em}

\usetikzlibrary{automata}
\usetikzlibrary{backgrounds}
\usetikzlibrary{fit}
\usetikzlibrary{positioning}

\usepackage{listings}
\renewcommand{\lstlistingname}{Listing}

\begin{document}

\title{Proposal for Theory of Tables}

\author{Andrew Reynolds \and Chenglong Wang}

\maketitle

\theoremstyle{definition}
\newtheorem{example}{Example}
\newcommand{\rem}[1]{\textcolor{red}{[#1]}}
\newcommand{\ajr}[1]{\rem{#1 --ajr}}
\newtheorem{definition}{Definition}

\global\long\def\autobox#1{#1}


\global\long\def\opname#1{\autobox{\operatorname{#1}}}


\global\long\def\dontcare{\_}


\global\long\def\bbracket#1{\left\llbracket #1\right\rrbracket }


\global\long\def\cnot#1{\centernot#1}






\global\long\def\roset#1{\left\{  #1\right\}  }


\global\long\def\ruset#1#2{\roset{#1\;\middle\vert\;#2}}


\global\long\def\union#1#2{#1\cup#2}


\global\long\def\bigunion#1#2{\bigcup_{#1}#2}


\global\long\def\intersection#1#2{#1\cap#2}


\global\long\def\bigintersection#1#2{\bigcap_{#1}#2}


\global\long\def\bigland#1#2{\bigwedge_{#1}#2}


\global\long\def\powerset#1{2^{#1}}




\global\long\def\cart#1#2{#1\times#2}


\global\long\def\tuple#1{\left(#1\right)}


\global\long\def\quoset#1#2{\left.#1\middle/#2\right.}


\global\long\def\equivclass#1{\left[#1\right]}


\global\long\def\refltransclosure#1{#1^{*}}




\newcommandx\funcapphidden[3][usedefault, addprefix=\global, 1=]{#2#1#3}


\global\long\def\funcapplambda#1#2{\funcapphidden[\,]{#1}{#2}}


\global\long\def\funcapptrad#1#2{\funcapphidden{#1}{\tuple{#2}}}


\global\long\def\funccomp#1#2{#1\circ#2}


\global\long\def\arrow#1#2{#1\to#2}


\global\long\def\func#1#2#3{#1:\arrow{#2}{#3}}




\global\long\def\N{\mathbb{N}}


\global\long\def\Z{\mathbb{Z}}


\global\long\def\Q{\mathbb{Q}}


\global\long\def\R{\mathbb{R}}


\global\long\def\D{\mathbb{D}}


\global\long\def\Zmod#1{\quoset{\Z}{#1\Z}}


\global\long\def\vector#1{\mathbf{#1}}


\global\long\def\dotprod#1#2{#1\cdot#2}


\global\long\def\true{\texttt{true}}


\global\long\def\false{\texttt{false}}






\global\long\def\strempty{\epsilon}




\global\long\def\kstar#1{#1^{*}}
\global\long\def\koption#1{#1^{?}}
\global\long\def\kstarn#1#2{#1^{#2}}


\global\long\def\kplus#1{#1^{+}}


\global\long\def\paren#1{\texttt{(}#1\texttt{)}}


\section{Theory of Tables}

This document describes a SMT-LIB theory of tables $\thtable$.
We provide its signature $\sigtable$ in two parts,
first describing its sorts and then its function symbols.
We assume that $\sigtable$ includes the symbols of the core theory, 
namely, the Boolean sort denoted $\syntax{Bool}$,
a binary polymorphic predicate $\syntax{=}$ denoting equality, for all sorts.

\section{Sort Symbols}

The signature $\sigtable$ contains 
the integer sort $\syntax{Int}$ as well as
the following three (parametric) sorts.

\subsection{Function Sorts}
The signature $\sigtable$ includes all sorts of the form:
\begin{align*}
\syntax{(->\ T_1\ ...\ T_k\ T)}
\end{align*}
for all sorts $\syntax{T_1, ..., T_k, T}$ where $k \geq 1$.
This sort denotes the class of functions taking arguments of sorts $\syntax{T_1, ..., T_k}$
and having return sort $\syntax{T}$.
Note that a predicate sort is a function sort whose return
sort is Boolean.

\begin{comment}
\paragraph{Notes}
\begin{itemize}
\item The term $\syntax{ 
(lambda\ ((x_1\ T_1)\ ...\ (x_n\ T_n))\ t)
}$
has sort 
$\syntax{(->\ T_1\ ...\ T_n\ T)}$
when $\syntax{t}$ has sort $\syntax{T}$.
\end{itemize}
\end{comment}

\subsection{Tuple Sorts}

The signature $\sigtable$ includes all sorts of the form:
\begin{align*}
\syntax{ (Tuple\ T_1\ ...\ T_k) }
\end{align*}
for all sorts $\syntax{T_1, ..., T_k}$.
This sort denotes tuples taking arguments of sorts $\syntax{T_1, ..., T_k}$.

\subsection{Bag Sorts}
The signature $\sigtable$ includes all sorts of the form:
\begin{align*}
\syntax{ (Bag\ T) }
\end{align*}
for all sorts $\syntax{T}$. 
This sort denotes bags (i.e. bags) of elements of sort $\syntax{T}$.
Note that bags allow duplication of elements and do not maintain an order
of the elements they contain.
The \emph{cardinality} of a bag is the number of elements it contains.
%An element may occur with any multiplicity in a bag.
%The elements of a bag are unordered.

\paragraph{Notes}
A table is bag whose elements are tuples.
We write $\syntax{ (Table\ T_1 ...\ T_k)}$ as shorthand for 
the sort $\syntax{ (Bag\ (Tuple\ T_1\ ...\ T_k)) }$ 
in the remainder of the document.
We refer to $\syntax{T_1}, \ldots, \syntax{T_k}$ as the \emph{columns} of table sort above.

\section{Function Symbols}

The signature $\sigtable$ includes function symbols for
lambda abstractions,
tuples, bags, and
three function symbols denoting table operations.
We list these in the following.
Function symbols in its signature may be polymorphic.
To provide the sort of (classes of) polymorphic functions,
we follow the convention for theory definitions in SMT-LIB. %~\cite{}.
In particular, the \emph{sort definition} for a polymorphic function $\syntax{f}$
has the following syntax:
\begin{verbatim}
(par (X_1 ... X_j) (f T_1 ... T_k T))
\end{verbatim}
Above, $\syntax{X_1\ ...\ X_j}$ are \emph{sort parameters}
that may freely occur in sorts $\syntax{T_1,\ ...\ T_k}$ and $\syntax{T}$.
Above, $\syntax{T_1,\ ...\ T_k}$ are the argument sorts of $\syntax{f}$
and $\syntax{T}$ is its return sort.

\subsection{Functions}
We assume
the symbol $\syntax{lambda}$ denotes lambda abstraction.
Its syntax is:
\begin{align*}
\syntax{ 
(lambda\ ((x_1\ T_1)\ ...\ (x_n\ T_n))\ t)
}
\end{align*}
where $\syntax{t}$ is a term,
and $\syntax{x_1\ ...\ x_n}$ are bound variables.
It has sort $\syntax{(->\ T_1\ ...\ T_n\ T)}$
when $\syntax{t}$ has sort $\syntax{T}$.
It denotes the anonymous function
returning $t$ for input arguments $\syntax{x_1}, \ldots, \syntax{x_n}$.
%The term 
%$\syntax{((lambda\ ((x_1\ T_1)\ ...\ (x_n\ T_n))\ t)\ s_1\ ...\ s_n)}$
%is interpreted as $t$ where $x_1, \ldots, x_n$ are replaced by
%$s_1, \ldots, s_n$.

\subsection{Tuples}

\subsubsection{Tuple Construction}
The signature $\sigtable$ includes 
the function symbol $\syntax{tuple}$.
Its sort definition is
\begin{verbatim}
(par (X_1 ... X_j) 
  (tuple
  
    ; The arguments of the tuple we are constructing
    X_1 ... X_j
    
    ; The return sort
    (Tuple X_1 ... X_j)
  )
)
\end{verbatim}
The term $\syntax{(tuple\ u_1\ ...\ u_n)}$ is
interpreted as the tuple comprised of 
the (ordered) list of elements $\syntax{u_1,\ ...\ u_n}$. 

\subsubsection{Tuple Selection}
The signature $\sigtable$ includes 
all indexed function symbols of the form
$\syntax{ 
(\_\ sel\ n)
}$
where $\syntax{n}$ is a numeral.
Its sort definition is
\begin{verbatim}
(par (X_1 ... X_j) 
  ((_ sel n)
  
    ; The tuple we are selecting from
    (Tuple X_1 ... X_j)
    
    ; The return sort
    X_n
  )
)
\end{verbatim}
where $1 \leq \syntax{n} \leq \syntax{j}$.
The term $\syntax{((\_\ sel\ n)\ t)}$ is
interpreted as the $\syntax{n}^{th}$ argument of the tuple $\syntax{t}$.

\subsubsection{Tuple Projection}
\label{sec:tup-project}
The signature $\sigtable$ includes 
all indexed function symbols of the form
$
\syntax{ 
(\_\ project\ n_1\ ...\ n_k)
}$
where $\syntax{n_1}, \ldots, \syntax{n_k}$ are numerals.
Its sort definition is:
\begin{verbatim}
(par (X_1 ... X_j) 
  ((_ project n_1 ... n_k)
  
    ; The table to project
    (Tuple X_1 ... X_j)
    
    ; The return sort
    (Tuple X_{n_1} ... X_{n_k})
  )
)
\end{verbatim}
where for each $i = 1, \ldots, j$,
$1 \leq n_i \leq j$.
The term
$\syntax{((\_\ project\ n_1\ ...\ n_k)\ t)}$
returns the tuple obtained by concatenating the
values from $t$ at positions $n_1 \ldots n_k$.
In particular, this term is equivalent to the
term $\syntax{(tuple\ ((\_\ sel\ n_1)\ t)\ ...\ ((\_\ sel\ n_k)\ t))}$.

\paragraph{Notes}
\begin{itemize}
\item
The function
$\syntax{project}$ (without indices)
denotes the operator above where $k=0$, which always returns the empty
tuple.
\end{itemize}

\subsection{Bags}

\subsubsection{Bag Construction}
The signature $\sigtable$ includes 
the function symbol $\syntax{bag}$ whose sort definition is
\begin{verbatim}
(par (X) 
  (bag
  
    ; The argument and multiplicity of the bag we are constructing
    X Int
    
    ; The return sort
    (Bag X)
  )
)
\end{verbatim}
The term $\syntax{(bag\ t\ n)}$ is
interpreted as the bag containing $n$ copies of the element $\syntax{t}$.
%where notice the order of these elements is irrelevant
%and $\syntax{t_i}$ may be equal to $\syntax{t_j}$ for any $\syntax{i},\syntax{j}$.
%If $n=0$,
%then $\syntax{bag}$ denotes the empty bag.
Notice that bags containing multiple distinct elements
can be constructed using $\syntax{union\_disjoint}$,
introduced in Section~\ref{sec:bag-compositions}

\subsubsection{Bag Count}
The signature $\sigtable$ includes 
the function symbol $\syntax{count}$ whose sort definition is
\begin{verbatim}
(par (X)
  (count 
  
    ; The element we are counting
    X 
    
    ; The bag in which the element may occur
    (Bag X)
    
    ; The return sort
    Int
  )
)
\end{verbatim}
The term $\syntax{(count\ u\ t)}$ is
interpreted as the number of times element $\syntax{u}$ occurs in $\syntax{t}$.

\paragraph{Notes}
\begin{itemize}
\item
We do not include an explicit membership predicate for bags, although note that
the formula $\syntax{(>\ (count\ u\ t)\ 0)}$ holds exactly when $\syntax{u}$
is a member of $\syntax{t}$ (with any multiplicity).
\end{itemize}

\subsubsection{Bag Compositions}
\label{sec:bag-compositions}
The signature $\sigtable$ includes 
the binary function symbols 
$\syntax{union\_max}$, $\syntax{union\_disjoint}$, 
$\syntax{intersect\_min}$, $\syntax{difference\_subtract}$ and
$\syntax{difference\_remove}$.
Each $\syntax{f}$ of the aforementioned operators
has sort definition:
\begin{verbatim}
(par (X) (f (Bag X) (Bag X) (Bag X)))
\end{verbatim}
These functions are such that for all elements $\syntax{x}$
and bags $\syntax{A,B}$:
\begin{verbatim}
(count x (union_max A B))            = (max (count x A) (count x B))
(count x (union_disjoint A B))       = (+ (count x A) (count x B))
(count x (intersect_min A B))        = (min (count x A) (count x B))
(count x (difference_subtract A B))  = (max (- (count x A) (count x B)) 0)
(count x (difference_remove A B))    = (ite (> (count x B) 0) 0 (count x A))
\end{verbatim}
where $\syntax{max}$ and $\syntax{min}$ are binary functions
returning the maximum (resp. minimum) of two integers, and $\syntax{ite}$
is the if-then-else construct.

\subsubsection{Bag Cardinality}
The signature $\sigtable$ includes 
the function symbol $\syntax{card}$ whose sort definition is
\begin{verbatim}
(par (X) 
  (card 
  
    ; The bag to process
    (Bag X) 
    
    ; The return sort
    Int
  )
)
\end{verbatim}
The term $\syntax{(card\ t)}$ is equal to
the total number of elements (taking into account multiplicity)
that occur in the bag $\syntax{t}$.
For example, the term $\syntax{(card (bag\ t\ n))}$ is
interpreted as $n$.

\subsubsection{Bag Filtering}
The signature $\sigtable$ includes 
the function symbol $\syntax{filter}$. Its sort definition is:

\begin{verbatim}
(par (X)
  (filter
  
    ; The predicate we are filtering based on
    (-> X Bool)
  
    ; The bag to filter
    (Bag X)
    
    ; The return sort
    (Bag X)
  )
)
\end{verbatim}

\paragraph{Semantics}
The filter operator keeps only the elements
from its first argument
for which the predicate of its second argument is true.
In other words,
$\syntax{(filter\ P\ t)}$
is the bag containing exactly
the elements $u$ from $\syntax{t}$
such that $\syntax{P}(u)$ is true.
Notice that if $\syntax{P}(u)$ is true,
then $u$ has the same multiplicity in $\syntax{(filter\ P\ t)}$
as it did in $\syntax{t}$.

\subsubsection{Bag Map}
The signature $\sigtable$ includes 
the function symbol $\syntax{map}$. Its sort definition is:

\begin{verbatim}
(par (X_1 X_2)
  (map
    
    ; The function we are applying
    (-> X_1 X_2)
  
    ; The bag to process
    (Bag X_1)
    
    ; The return sort
    (Bag X_2)
  )
)
\end{verbatim}

\paragraph{Semantics}
The map operator runs the function provided
in the second argument to all elements in the bag.
In other words,
$\syntax{(map\ f\ t)}$
is the bag containing exactly
the elements $\syntax{(f\ u)}$
for each $\syntax{u}$ that occurs in $\syntax{t}$.
The multiplicity of $\syntax{(f\ u)}$
in $\syntax{(map\ f\ t)}$
is the same as the multiplicity of $\syntax{u}$ in $\syntax{t}$.

\subsubsection{Bag Fold}
\label{sec:fold}
The signature $\sigtable$ includes 
the function symbol $\syntax{fold}$. Its sort definition is:

\begin{verbatim}
(par (X_1 X_2)
  (fold
    
    ; The "combining" function we are applying
    (-> X_1 X_2 X_2)
    
    ; The initial value of the fold
    X_2
  
    ; The bag to process
    (Bag X_1)
    
    ; The return sort
    X_2
  )
)
\end{verbatim}


\paragraph{Semantics}
Given a bag $t$ whose elements are $u_{1}, \ldots, u_{n}$,
assume an arbitrary total ordering over these elements $\preceq$.
Then, $\syntax{(fold\ c\ i\ t)}$ is equivalent to:
\begin{verbatim}
(c u_n ... (c u_2 (c u_1 b)) ... )
\end{verbatim}
In other words, given a start value $b$,
we apply the combining function $c$, taking these elements
as first arguments in order.

\paragraph{Notes}
\begin{itemize}
\item
The semantics above impose no restrictions on the ordering $\preceq$,
which determines in which the tuples $\{ u_{i1}, \ldots, u_{in} \}$
are passed to the combining function $c$.
Moreover, 
we do not provide syntax to specify such an ordering.
Instead, 
we remark that the above semantics
do not require an ordering when the combining function is
\emph{order-agnostic},
that is, for all $b, v_1, \ldots, v_n$,
we have that:
\begin{verbatim}
(c v_1 ... (c v_n b) ... ) = (c w_1 ... (c w_n b) ... )
\end{verbatim}
for all permutations $( w_1, \ldots, w_n )$ of $( v_1, \ldots, v_n )$.
If the function $c$ is order-agnostic,
then each $r_i$ in the above computation will be the same regardless
of the ordering $\preceq$.
Thus, as stated, the above semantics assume that $c$ is order-agnostic.
The semantics
of aggregation functions $A$ over
combining functions that are not order-agnostic are undefined.
%Thus, we assume that the operator $c$ is
%associatiative and commutative.
\end{itemize}

\subsubsection{Bag Partition}
The signature $\sigtable$ includes 
the function symbol $\syntax{partition}$. Its sort definition is:

\begin{verbatim}
(par (X)
  (partition
    
    ; The equivalence relation
    (-> X X Bool)
  
    ; The bag to process
    (Bag X)
    
    ; The return sort
    (Bag (Bag X))
  )
)
\end{verbatim}

\paragraph{Semantics}
Given a bag $t$, 
the term $\syntax{(partition\ r\ t)}$ is interpreted
as a bag of bags that partition the elements of $\syntax{t}$,
whose elements (of bag sort) are $\syntax{t_1}, \ldots, \syntax{t_n}$.
For each pair of elements $\syntax{u_i}, \syntax{u_j}$
occurring in $\syntax{t_k}, \syntax{t_l}$ respectively,
we have that $k=l$ if and only if $\syntax{(r\ u_i\ u_j)}$ is true.

\paragraph{Notes}
\begin{itemize}
\item
We expect that the argument $r$ passed as the
first argument to $\syntax{partition}$ is an equivalence relation.
In particular, this means it is reflexive, symmetric and transitive.
Otherwise, $\syntax{partition}$ applied to that relation is undefined.
\item
If $\syntax{u}$ occurs with some multiplicity $n$ in $\syntax{t}$,
then $\syntax{u}$ occurs with the same multiplicity $n$ in some $\syntax{t_i}$.
\end{itemize}

\subsection{Tables}

\subsubsection{Table Projection}
The signature $\sigtable$ includes 
all indexed functions symbols of the form
\begin{align*}
\syntax{ 
(\_\ project\ n_1\ ...\ n_k)
}
\end{align*}
where $n_1, \ldots, n_k$ are numerals.
Note this operator is overloaded,
as a separate projection function (of the same name)
is also defined for tuples in Section~\ref{sec:tup-project}.
Its sort definition is:
\begin{verbatim}
(par (X_1 ... X_j) 
  ((_ project n_1 ... n_k)
  
    ; The table to project
    (Table X_1 ... X_j)
    
    ; The return sort
    (Table X_{n_1} ... X_{n_k})
  )
)
\end{verbatim}
where for each $i = 1, \ldots, j$,
$1 \leq n_i \leq j$. 
Note that
$\syntax{project}$ without indices
denotes the operator above where $k=0$.

\paragraph{Semantics}

A projection operator keeps (copies) of the given columns
in each (tuple) element of its argument.
In other words,
let $p$ be the projection operator denoted by $\syntax{(\_\ project\ n_1\ ...\ n_k)}$
and let $t$ be a table.
Then, $p(t)$
is the table containing a tuple of the form
$(u_{n_1}, \ldots, u_{n_k})$
for each tuple $(u_1, \ldots, u_j)$ from $t$.

\paragraph{Notes}
\begin{itemize}
\item
The term $\syntax{((\_\ project\ n_1\ ...\ n_k)\ t)}$
can be seen as syntax sugar for $\syntax{(map\ p\ t)}$ where $\syntax{p}$
is a function that
applies the tuple projection on tuples $\syntax{(Tuple\ X_1\ ...\ X_j)}$
to obtain tuples $\syntax{(Tuple\ X_{n_1}\ ...\ X_{n_k})}$:
\begin{verbatim}
(lambda ((t (Tuple X_1 ... X_j))) ((_ project n_1 ... n_k) t))
\end{verbatim}
\item
It may be the case that $\syntax{n_i} = \syntax{n_j}$ for some $i \neq j$,
in which case two (or more) copies of the column at that position
are in the result of the projection.
\item
The cardinality of $\syntax{((\_\ project\ n_1\ ...\ n_k)\ t)}$ is
same as the cardinality of $\syntax{t}$.
\end{itemize}

\subsubsection{Table Join}
The signature $\sigtable$ includes 
all indexed functions symbols of the form
\begin{align*}
\syntax{ 
(\_\ join\ n_1\ m_1\ ...\ n_k\ m_k)
}
\end{align*}
where $n_1, m_1, \ldots, n_k, m_k$ are numerals.
Its sort definition is:

\begin{verbatim}
(par (X_1 ... X_j Y_1 ... Y_l)
  ((_ join n_1 m_1 ... n_k m_k)
  
    ; The tables to join
    (Table X_1 ... X_j)
    (Table Y_1 ... Y_j)
    
    ; The return sort
    (Table X_1 ... X_j Y_1 ... Y_j)
  )
)
\end{verbatim}
where for all $i = 1, \ldots k$,
we have that $1 \leq \syntax{n_i} \leq \syntax{j}$ and $1 \leq \syntax{m_i} \leq \syntax{l}$. 
Note that
$\syntax{join}$ (without numeral indices) denotes the above operator where $\syntax{k}$ is $0$.

\paragraph{Semantics}
When $\syntax{k}=0$, we have that $\syntax{(join\ t_1\ t_2)}$
denotes the table containing a tuple corresponding to the concatentation
of tuples $u_1$ and $u_2$
for each $u_1 \in \syntax{t_1}$ and $u_2 \in \syntax{t_2}$.
When $\syntax{k}>0$, $\syntax{((\_\ join\ n_1\ m_1\ ...\ n_k\ m_k)\ t_1\ t_2)}$
is syntax sugar for $\syntax{(filter\ P\ (join\ t_1\ t_2))}$ where $\syntax{P}$ is
the predicate:
\begin{verbatim}
(lambda ((t_1 (Tuple X_1 ... X_j)) (t_2 (Tuple Y_1 ... Y_l)))
  (= ((_ project n_1 ... n_k) t_1) ((_ project m_1 ... m_k) t_2))
)
\end{verbatim}
where $\syntax{o_i}$ is the numeral $\syntax{j+m_i}$ for each $i = 1, \ldots, k$.
In other words,
the join keeps only concatenations of tuples $u_1 \in \syntax{t_1}$ and
$u_2 \in \syntax{t_2}$ 
that match on each of the pairs of argument indices (columns) specified by
the indices of the operator,
where $n_1\ ...\ n_k$ are indices corresponding to arguments of $u_1$
and $m_1\ ...\ m_k$ are indices corresponding to arguments of $u_2$.

\paragraph{Notes}
\begin{itemize}
\item The cardinality of $\syntax{(join\ t_1\ t_2)}$
is the cardinality of $\syntax{t_1}$ times the cardinality of $\syntax{t_2}$.
\end{itemize}


\subsubsection{Table Aggregation}
The signature $\sigtable$ includes 
all indexed function symbols of the form
$
\syntax{ 
(\_\ aggr\ n_1\ ...\ n_k)
}
$
where $\syntax{n_1}, \ldots, \syntax{n_k}$ are numerals,
called \emph{table aggregation operations}. Their sort definition is:

\begin{verbatim}
(par (X_1 ... X_j X) 
  ((_ aggr n_1 ... n_k)
    
    ; The "combining function"
    (-> (Tuple X_1 ... X_j) X X)
    
    ; The "initial value"
    X
    
    ; The table to aggregate
    (Table X_1 ... X_j)
    
    ; The return sort
    (Bag X)
  )
)
\end{verbatim}
where for all $i = 1, \ldots k$,
we have that $1 \leq \syntax{n_i} \leq \syntax{j}$. 

\paragraph{Semantics}
Let $\syntax{((\_\ aggr\ n_1\ ...\ n_k)\ c\ i\ t)}$
be a well-sorted term for
combining function $\syntax{c}$, 
initial value $\syntax{i}$,
and table-to-aggregate $\syntax{t}$.
Assume that $c$ is \emph{order-agnostic} (see Section~\ref{sec:fold}).
The term $\syntax{((\_\ aggr\ n_1\ ...\ n_k)\ c\ i\ t)}$ is
the union of elements computed via a fold operation involving $i$, $c$, and $f$
that run over elements in a partition of $t$.
In particular, this application of an aggregation function is syntax sugar for:
\begin{verbatim}
(map (fold c i) (partition eq t))
\end{verbatim}
where $\syntax{eq}$ is the relation over tuples that holds
when tuples are equivalent for indices $\syntax{n_1}, \ldots, \syntax{n_k}$:
\begin{verbatim}
(lambda ((t_1 (Tuple X_1 ... X_j)) (t_2 (Tuple X_1 ... X_j)))
  (= ((_ project n_1 ... n_k) t_1) ((_ project n_1 ... n_k) t_2))
)
\end{verbatim}
Above, notice that 
$\syntax{(fold\ c\ i)}$ is a partial application of the fold
function which takes a $\syntax{(Bag\ (Tuple\ X_1\ ...\ X_j))}$
and returns a term of type $\syntax{X}$.

\begin{comment}
In detail,
let $\{ t_1, \ldots, t_n \}$ be a partition of table $t$
where for each table $t_i$, each of its (tuple) elements
are identical for arguments $n_1, \ldots, n_k$.
%In other words, 
%all elements of $\syntax{((\_\ project\ n_1\ ...\ n_k) t_i)}$
%are identical.
For each $i = 1, \ldots, n$,
let $M_i$ be the bag $p(t_i)$
where $p$ is the projection operator denoted by $\syntax{(\_\ project\ n)}$.
For each such bag $M_i = \{ u_{i1}, \ldots, u_{in} \}$,
assume an arbitrary ordering over these elements $\preceq$.
Let $r_i$ be the term:
\begin{align*}
f( M_i, c( u_{in}, \ldots, c( u_{i1}, b ) \ldots ) )
\end{align*}
where $u_{i1} \preceq \ldots \preceq u_{in}$.
%which is a term whose type is that of the $n^{th}$ column of the type of $t$.
In other words,
$r_i$ is the result of a fold operation
starting with \emph{initial value} $b$,
applying \emph{combining function} $c$ to
the accumulated value and each of the tuples $u_{i1}, \ldots, u_{in}$,
and finally applying a \emph{finalization} function $f$
to the bag $M_i$ and to the result of the aforementioned computation.
Then,
$A( t, b, c, f)$ is
the bag union of (single argument) tuples $\{ (r_1), \ldots, (r_n) \}$.

\paragraph{Unspecified Finalize Function}
We overload the aggregation function
such that its fourth argument is optional.
If the fourth argument is not specified,
then 
$
\syntax{
((\_\ aggr\ n_1\ ...\ n_k\ n)\ t\ b\ c)
}
$
is syntax sugar for
$
\syntax{
((\_\ aggr\ n_1\ ...\ n_k\ n)\ t\ b\ c\ ident)
}
$
where $\syntax{ident}$ is the function
$\syntax{(lambda\ ((y\ (Table\ T_1 ...\ T_k))\ (x\ X_a))\ x)}$.
In other words, this function returns the result of the
fold computation described above unchanged.
\end{comment}

\section{Examples}



\end{document}
